\vspace{0.5em}
\begin{thebibliography}{99}

\bibitem{c1}\label{ref:tetrisGame} https://en.wikipedia.org/wiki/Tetris

\bibitem{c2}\label{ref:tetrisRepo} https://github.com/brenns10/tetris

\bibitem{c3}\label{ref:CilkPlus} https://cilkplus.github.io

\bibitem{c4}\label{ref:tetrisImage} https://tetris.fandom.com/wiki/SRS?file=SRS-pieces.png

\bibitem{c5}\label{ref:Genetic} Man, Kim-Fung, Kit-Sang Tang, and Sam Kwong. "Genetic algorithms: concepts and applications [in engineering design]." IEEE transactions on Industrial Electronics 43.5 (1996): 519-534.

\bibitem{c6}\label{ref:Minimax} Korf, Richard E., and David Maxwell Chickering. "Best-first minimax search." Artificial intelligence 84.1-2 (1996): 299-337.

\bibitem{c7}\label{ref:CilkReducer} Frigo, Matteo, et al. "Reducers and other Cilk++ hyperobjects." Proceedings of the twenty-first annual symposium on Parallelism in algorithms and architectures. 2009.

\bibitem{c8}\label{ref:TetrisHard} Demaine, Erik D., Susan Hohenberger, and David Liben-Nowell. "Tetris is hard, even to approximate." International Computing and Combinatorics Conference. Springer, Berlin, Heidelberg, 2003.

\bibitem{c9}\label{ref:MonteCarlo} Galván-López, Edgar, et al. "Heuristic-based multi-agent monte carlo tree search." IISA 2014, The 5th International Conference on Information, Intelligence, Systems and Applications. IEEE, 2014.

\bibitem{c10}\label{ref:AlphaBeta} Singhal, Shubhendra Pal, and M. Sridevi. "Comparative study of performance of parallel Alpha Beta Pruning for different architectures." 2019 IEEE 9th International Conference on Advanced Computing (IACC). IEEE, 2019.

\bibitem{c11}\label{ref:Burgiel} Burgiel, Heidi. "How to lose at Tetris." The Mathematical Gazette 81.491 (1997): 194-200.

\begin{comment}

\bibitem{c1} G. O. Young, ÒSynthetic structure of industrial plastics (Book style with paper title and editor),Ó 	in Plastics, 2nd ed. vol. 3, J. Peters, Ed.  New York: McGraw-Hill, 1964, pp. 15Ð64.
\bibitem{c2} W.-K. Chen, Linear Networks and Systems (Book style).	Belmont, CA: Wadsworth, 1993, pp. 123Ð135.
\bibitem{c3} H. Poor, An Introduction to Signal Detection and Estimation.   New York: Springer-Verlag, 1985, ch. 4.
\bibitem{c4} B. Smith, ÒAn approach to graphs of linear forms (Unpublished work style),Ó unpublished.
\bibitem{c5} E. H. Miller, ÒA note on reflector arrays (Periodical styleÑAccepted for publication),Ó IEEE Trans. Antennas Propagat., to be publised.
\bibitem{c6} J. Wang, ÒFundamentals of erbium-doped fiber amplifiers arrays (Periodical styleÑSubmitted for publication),Ó IEEE J. Quantum Electron., submitted for publication.
\bibitem{c7} C. J. Kaufman, Rocky Mountain Research Lab., Boulder, CO, private communication, May 1995.
\bibitem{c8} Y. Yorozu, M. Hirano, K. Oka, and Y. Tagawa, ÒElectron spectroscopy studies on magneto-optical media and plastic substrate interfaces(Translation Journals style),Ó IEEE Transl. J. Magn.Jpn., vol. 2, Aug. 1987, pp. 740Ð741 [Dig. 9th Annu. Conf. Magnetics Japan, 1982, p. 301].
\bibitem{c9} M. Young, The Techincal Writers Handbook.  Mill Valley, CA: University Science, 1989.
\bibitem{c10} J. U. Duncombe, ÒInfrared navigationÑPart I: An assessment of feasibility (Periodical style),Ó IEEE Trans. Electron Devices, vol. ED-11, pp. 34Ð39, Jan. 1959.
\bibitem{c11} S. Chen, B. Mulgrew, and P. M. Grant, ÒA clustering technique for digital communications channel equalization using radial basis function networks,Ó IEEE Trans. Neural Networks, vol. 4, pp. 570Ð578, July 1993.
\bibitem{c12} R. W. Lucky, ÒAutomatic equalization for digital communication,Ó Bell Syst. Tech. J., vol. 44, no. 4, pp. 547Ð588, Apr. 1965.
\bibitem{c13} S. P. Bingulac, ÒOn the compatibility of adaptive controllers (Published Conference Proceedings style),Ó in Proc. 4th Annu. Allerton Conf. Circuits and Systems Theory, New York, 1994, pp. 8Ð16.
\bibitem{c14} G. R. Faulhaber, ÒDesign of service systems with priority reservation,Ó in Conf. Rec. 1995 IEEE Int. Conf. Communications, pp. 3Ð8.
\bibitem{c15} W. D. Doyle, ÒMagnetization reversal in films with biaxial anisotropy,Ó in 1987 Proc. INTERMAG Conf., pp. 2.2-1Ð2.2-6.
\bibitem{c16} G. W. Juette and L. E. Zeffanella, ÒRadio noise currents n short sections on bundle conductors (Presented Conference Paper style),Ó presented at the IEEE Summer power Meeting, Dallas, TX, June 22Ð27, 1990, Paper 90 SM 690-0 PWRS.
\bibitem{c17} J. G. Kreifeldt, ÒAn analysis of surface-detected EMG as an amplitude-modulated noise,Ó presented at the 1989 Int. Conf. Medicine and Biological Engineering, Chicago, IL.
\bibitem{c18} J. Williams, ÒNarrow-band analyzer (Thesis or Dissertation style),Ó Ph.D. dissertation, Dept. Elect. Eng., Harvard Univ., Cambridge, MA, 1993. 
\bibitem{c19} N. Kawasaki, ÒParametric study of thermal and chemical nonequilibrium nozzle flow,Ó M.S. thesis, Dept. Electron. Eng., Osaka Univ., Osaka, Japan, 1993.
\bibitem{c20} J. P. Wilkinson, ÒNonlinear resonant circuit devices (Patent style),Ó U.S. Patent 3 624 12, July 16, 1990. 
\end{comment}


\end{thebibliography}